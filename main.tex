\documentclass{article}
\usepackage[utf8]{inputenc}
\usepackage{amsmath} % załatwiło problem
\title{egzamin rp}
\author{supp.crossbooking }
\date{June 2019}
\usepackage{polski}
\usepackage[a4paper,margin=1.75cm]{geometry}
\usepackage{amsmath}
\usepackage{amsthm}
\usepackage{amsfonts}
\usepackage{pgfplots}
\usepackage{hyperref}
\usepackage{todonotes}
\usepackage{thmtools}
% Symbol normy
\newcommand{\norm}[1]{\left\lVert#1\right\rVert}
 
% Paragrafy bez wcięć
\setlength\parindent{0pt}

\setlength{\parskip}{6pt}

% Style dla twierdzeń itd.
\theoremstyle{plain}
\newtheorem*{theorem}{Twierdzenie}

\theoremstyle{definition}
\newtheorem*{definition}{Definicja}
\newtheorem*{corollary}{Wniosek}

\theoremstyle{remark}
\newtheorem*{remark}{Uwaga}

\newcommand{\abs}[1]{\left|#1\right|} % Absolute value
\newcommand{\norm}[1]{\left\lVert#1\right\rVert} % Norm
\newcommand{\closure}[1]{\overline{#1}} % Set closure
\newcommand{\extcomplex}{\overline{\mathbb{C}}} % Extended complex plane
\newcommand{\eqtext}[1]{\overset{\text{\tiny\sffamily #1}}{=}} % Text over equality sign
\newcommand{\Forall}[1]{\mathop{\vcenter{\hbox{\LARGE$\forall$}}}\limits_{#1}} % Huge \forall for full-line rendering
\newcommand{\res}{\mathop{\text{Res}}\limits} % Residuum
%\setlist[enumerate]{label=(\alph*),ref=(\alph*),leftmargin=*,topsep=0.4ex,itemsep=-0.6ex,partopsep=1ex,parsep=1ex} % List margins
%\setlist[itemize]{topsep=0.4ex,itemsep=-0.6ex,partopsep=1ex,parsep=1ex} % List margins

\begin{document}

\maketitle
{\small\listoftheorems[ignoreall,onlynamed={theorem,definition,remark,corollary}]}

\section{Introduction}
Co było na wykładzie?
\subsection{1 wykład}
\begin{theorem}{Zasadniczne twierdzenie algebry FAN}
Każdy wieloman zespolony ma pierwiastek zespolony 
\end{theorem}

\subsection{2 wykład}
\begin{definition}[Obszar]
Niepusty i spójny podzbiór płaszczyzny zespolonej
\end{definition}

\begin{definition}{Obszar jednospójny FAN2012, FAN2009}
Obszar $U \subset \mathbb{C}$ jest obszarem jednospójnym, jeśli każdą zawartą w nim pętlę można w sposób ciągły zdeformować do punktu
\end{definition}

\begin{remark}
Brzeg takiego obszaru ma wtedy jedną składową spójności.
\end{remark}

\begin{itemize}
    \item Droga
    \item Spójna topologiczne
    \item klasa abstrakcji, łukowo spójność
\end{itemize}{}

\begin{definition}[różniczkowalność w sensie zespolonym]

\end{definition}{}

\begin{definition}[Holomorficzność]

\end{definition}{}

\begin{definition}[Rówanie Cauchego Riemana]

\end{definition}

\begin{definition}[Koforemntość]

\end{definition}{}

\begin{theorem}[Twierdzenie Riemanna o przekształceniach konforemnych]

\end{theorem}

\begin{definition}[Całka krzywoliniowa Riemana po drodze FAN2019-2]

\end{definition}

\todo{Czy zachodzi wzór?}

\subsection{3 wykład 28-30}

\begin{theorem}[Gauss-Lucas str14]


\end{theorem}

\begin{definition}{Całkowanie po krzywych]

\end{definition}



\begin{theorem}{Tw Laurenta}

\end{theorem}



\subsection{Wykład 5}

\begin{itemize}
    \item Fakt 1
    \item Fakt 2
    
\end{itemize}

\begin{theorem}{Lemat Całkowy Goursata FAN2012}

\end{theorem}{}

\begin{theorem}{Tw. Cauchego o krzywych homotopijnych FAN2012}

\end{theorem}

\subsection{Wykład 25.11.2019{24}}


\begin{theorem}[Wzór Całkowy Cauchego 101]
Niech $ \closure{D} \subset U \subset \mathbb{C} $, gdzie $ D $, $ U $ dowolne obszary.
  Niech $ f $ holomorficzna na $ U $.
  Wtedy
  \[
  f(z) = \frac{1}{2 \pi i} \int_{\partial D} \frac{f(\zeta)}{(\zeta - z)} d\zeta 
  \]
\end{theorem}

\begin{theorem}[Rozwijanie funkcji holomorficznej w szereg Taylora 104]
  Niech $ f $ będzie holomorficzna w obszarze $ U $ oraz $ K(a, R) \subset D $. Wówczas istnieje szereg postaci $ \sum_{n = 0}^{\infty} c_{n}(z-a)^{n} $, zbieżny do f(z) wewnątrz koła $ K(a, R) $. Ten szereg nazywamy szeregiem Taylora.
\end{theorem}

\begin{proof}
  Całkujemy po koncentrycznych okręgach wewnątrz koła $ K(a, R) $.
  \begin{align*}
    f(z) &
    = \frac{1}{2 \pi i} \int_{C(a, r)} \frac{f(\zeta)}{\zeta-z} d\zeta
    = \frac{1}{2 \pi i} \int_{C(a, r)} \frac{f(\zeta)}{\zeta-a-(z-a)} d\zeta \\ &
    = \frac{1}{2 \pi i} \int_{C(a, r)} \frac{f(\zeta)}{\zeta-a} \cdot \frac{1}{1 - \frac{z-a}{\zeta-a}} d\zeta
    = \frac{1}{2 \pi i} \int_{C(a, r)} \frac{f(\zeta)}{\zeta-a} \cdot \sum_{n=0}^{\infty} \frac{(z-a)^{n}}{(\zeta-a)^{n}} d\zeta = (*)
  \end{align*}
  Korzystamy z kryterium Weierstrassa. Zauważmy, że
  $
    \abs{\frac{f(\zeta)}{\zeta-a} \cdot \frac{(z-a)^{n}}{(\zeta-a)^{n}}}
    < \frac{M}{r} \cdot \abs{\frac{z-a}{r}}^n
    < \frac{M}{r} \cdot 1
  $
  . Możemy więc całkować wyraz po wyrazie.
  \begin{align*}
    (*) &
    \eqtext{W} \sum_{n=0}^{\infty} \left( \frac{1}{2 \pi i} \int_{C(a, r)} \frac{f(\zeta)}{(\zeta-a)^{n+1}} d\zeta \right)(z-a)^{n}
    = \sum_{n = 0}^{\infty} c_{n}(z-a)^{n}
  \end{align*}
  Współczynniki $ c_{n} $ są następującej postaci:
  $$
    c_{n} = \frac{1}{2 \pi i} \int_{C(a, r)} \frac{f(\zeta)}{(\zeta-a)^{n+1}} d\zeta
  $$
\end{proof}

\begin{corollary}[Nierówność Cauchy'ego FAN2019-2]
  Niech $ M(r) = \max_{\abs{z-a} = r < R} \abs{f(z)} $. Wówczas zachodzi nierówność
  $$ \abs{c_{n}} = \abs{ \frac{1}{2 \pi i} \int_{C(a, r)} \frac{f(\zeta)}{(\zeta-a)^{n+1}} d\zeta } \leq \frac{1}{2 \pi} \cdot \frac{M(r)}{r^{n+1}} \cdot 2 \pi r = \frac{M(r)}{r^n} $$
\end{corollary}


\begin{theorem}{Transformata całkowa Cauchego}

\end{theorem}

\begin{theorem}{Lemat Schwarza}
Niech $ f $ funkcja holomorficzna w kole $ K(0, R) $.
  Jeśli $ f(0) = 0 $ oraz $ \forall z \in D\enskip\abs{f(z)} \leq M $, to
  $$
    \abs{f'(0)} \leq \frac{M}{R}
    \qquad \textnormal{oraz} \qquad
    \abs{f(z)} \leq \frac{M}{R}\abs{z}
  $$
  Ponadto, równość zachodzi wtedy i tylko wtedy, gdy $ f $ jest postaci
  $$
    f(z) = \frac{M}{R}e^{it}z
  $$
\end{theorem}


\begin{proof}
  Bez utraty ogólności możemy dowodzić dla $ M = 1 $, $ R = 1 $.

  Z $ f(0) = 0 \Rightarrow f(z) = z \cdot g(z) $, gdzie $g$ jest holomorficzna w kole $ K(0,1) $.
  Stosujemy zasadę maksimum dla $ g(z) $, $ r < R = 1 $.
  $$
    \max_{z \in \closure{K(0, r)}} \abs{g(z)} =
    \max_{z \in C(0, r)} \abs{g(z)}
    \leq \max_{z \in C(0, r)} \frac{\abs{f(z)}}{\abs{z}}
    \leq \frac{1}{r}
  $$
  Przy $ r \to 1 $ dostajemy $ max_{z \in \closure{K(0, r)}} \abs{g(z)} \leq 1 $.
  Stąd
  $$
    \Forall{z \in K(0, r)} 1 \geq \abs{g(z)}
    = \frac{\abs{f(z)}}{\abs{z}}
    \qquad \implies \qquad
    \abs{f(z)} \leq \abs{z}
  $$
  co dowodzi jednej nierówności z twierdzenia.
  Dla pochodnej mamy
  $$
    f(z) = zg(z)
    \quad\Rightarrow\quad
    f'(z) = g(z) + zg'(z)
    \quad\Rightarrow\quad
    f'(0) = g(0)
    \quad\Rightarrow\quad
    \abs{f'(0)} = \abs{g(0)}
    \leq 1
  $$

  Jeśli $ \abs{f(z_0)} = \abs{z_0} $ dla pewnego $ z_0 \neq 0 $, to $ \abs{g(z_0)} = 1 $.
  Czyli $ g $ osiąga maksimum wewnątrz $ K(0, 1) $, więc z zasady maksimum jest stała.
  Skoro $ \abs{g(z)} = 1 $, to $ g(z) = e^{it} $.
\end{proof}

\begin{theorem}[Wzór całkowy Cauchy'ego dla pochodnych FAN2019-2]
  Niech $ \closure{D} \subset G \subset \mathbb{C} $, gdzie $ D $, $ G $ dowolne obszary.
  Niech $ f $ holomorficzna na $ G $.
  Wtedy
  $$ \forall{n \in \mathbb{N}_0} \forall{a \in D} \quad f^{(n)}(a) = \frac{n!}{2 \pi i} \int_{\partial D} \frac{f(\zeta)}{(\zeta - a)^{n+1}} d\zeta $$
\end{theorem}

\begin{theorem}{Własności wartości Średniej}
\end{theorem}

\begin{theorem}[Zasada maksimum 26]
Niech $ f $ funkcja holomorficzna w obszarze $ D $.
  Jeśli $ \max_D \abs{f(z)} $ jest osiągane wewnątrz $ D $, to $ f $ jest stała.
\end{theorem}

\begin{theorem}{tw Louviella FAN2019-2}
 Każda funkcja całkowita i ograniczona jest stała.
\end{theorem}


\begin{proof}
 \begin{align*}
 f^{(n)}(0) = \frac{n!}{2 \pi i} \int_{\partial D} \frac{f(w)}{(w)^{n+1}} dw
 \end{align*}
 \[
   |f^{(n)}(0)| \leq \frac{n!}{2 \pi i} L(\partialD(0,R) \frac{\supp|f(w))}{R^n+1} dw \mapsto 0
 \]
\end{proof}

\begin{theorem}{zasada maksimum 2}

\end{theorem}

\subsection{Wykład}

\begin{theorem}{Zasada izolowanych zer 33}
Zero skończonej krotności $f$ holomorficznej jest izolowane, tzn nie jest punktem skupienia zbioru pozostałych zer.
\end{theorem}

\begin{theorem}{Zasada identyczności 34}
$U $ obszar $f, g \in H(U)$. Jeżeli $\{z \in \mathbb{C} f(z) = g(z) \}$ ma punkt skupienia należący do U to 
\[
\forall{z\in U} f(z) = g(z)
\]
\end{theorem}

\begin{theorem}{Tw Morery FAN2009}
Niech $f:U \mapsto \mathbb{C}$ jest ciągła. Wówczas następujące warunki są równoważne:
\end{theorem}
\begin{enumerate}
    \item $f$ jest holomorficzna na $U$
    \item $\int_{\partial \bigtriangleup} f(z)dz = 0 \quad \forall{\bigtriangleup} \subset U$
\end{enumerate}{}

\begin{proof}
 \begin{enumerate}
     \item $f$ jest holomorficzna na $U \implies$ f ma funkcje pierwotną $\implies \Forall{\bigtriangleup \subset U} \int_{\partial \bigtriangleup} f(z)dz = 0$
     \item \todo{na karce w drugą strone}
 \end{enumerate}
\end{proof}

\subsection{Wykład 9.12.2019- 38}

\begin{itemize}
    \item Fakt 1
    \item Fakt 2
\end{itemize}

\begin{theorem}{Zasada symetrii Riemanna Scharza}
Jeżeli $f^{+}$ i $f^{-}$ są holomoriczne w $\quad \Omega^{+}$ and $\Omega^{-}$ respectively, that extend continuously to I and
\[
f^{+}(x)=f^{-}(x) \quad \text { for all } x \in I
\]
then the function $f$ defined on $\Omega$ by
\[
f(z)=\left\{\begin{array}{ll}
{f^{+}(z)} & {\text { if } z \in \Omega^{+}} \\
{f^{+}(z)=f^{-}(z)} & {\text { if } z \in I} \\
{f^{-}(z)} & {\text { if } z \in \Omega^{-}}
\end{array}\right.
\]
is holomorphic on all of \Omega.
\end{theorem}

\begin{theorem}{Ciągła gałąź logarytmu}
                        
\end{theorem}

\todo{Zadanie domowe}

\subsection{Wyklad }
\begin{theorem}{Gałąź logarytmu}

\end{theorem}

\begin{theorem}{FAN2012}
Udowodnić, że każda funkcja holomorficzna, określona w obszarze jednospójnym i nie przyjmująca wartości 0, ma w nim gałąź argumentu, logarytmu i pierwiastka kwadratowego.
\end{theorem}

\begin{definition}{indeks Pętla FAN2009}

\end{definition}

\todo{Zadanie domowe}

\subsection{Wykład po 13.01.2020}

\begin{theorem}{Weierstrass FAN2009}

  Niech szereg funkcji holomorficznych (na $U$) $ f_n(z) $ będzie niemal jednostajnie zbieżny na obszarze $U$ do $f$. Wówczas $f$ jest holomorficzna na $U \subset \mathbb{C}$, a także $f_n(z)^{'}$ zbiega niemal jednostajnie do $f'$ na $U$.
  
  
\end{theorem}

\begin{remark}{Przypadek rzeczywisty}
$f_n(x) = \frac{1}{n}\sin(n^{2}x), f=0$ $f_n$ zbiega jednostajnie do f, jednak pochodne nie zbiegają $f_n^{'} = n \cos(n^2x)$ nie zbiegają jednostajnie do $f'(0)$
\end{remark}

\url{http://duch.mimuw.edu.pl/~pol/Funkcje%20Analityczne/Wyklady7,8}

\begin{theorem}{Reguła D' Hospitala}
f, g holomorficzna na otoczeniu otwartym $z_0$ i $f(z_0) = g(z_0) = 0$ i $g'(z_0) != 0$
\end{theorem}

\begin{theorem}{Carl-Weistrass}

\end{theorem}

\begin{theorem}{Tw. Laurent  o rozwijaniu w szreg}

\end{theorem}


\begin{theorem}{Tw o Residuach}

\end{theorem}

\begin{theorem}{DW-TW o Residuach}

\end{theorem}

\begin{theorem}{tw Riemmana}

\end{theorem}
\todo{Dlaczego nie można tutaj stosować tw. Morery FAN2012, FAN2009}

\begin{definition}{Funkcja meromorficzna FAN}

\end{definition}

\todo{Meromoriczna na C i nie jest stała. Wykaż, że jej obraz jest gęsty w C}

\begin{theorem}{Casoratiego-Sochockiego-Weierstrassa FAN2012}

\end{theorem}

\end{document}
